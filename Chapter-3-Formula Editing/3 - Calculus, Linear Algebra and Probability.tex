\documentclass[12pt]{article}
\usepackage{amsmath}
\usepackage{mathtools} % 矩阵环境
\usepackage{amssymb} % 数学符号扩展

\begin{document}
%% ===== 微积分 ======================================== %%
%% 极限
\[ 
\lim_{x\to_\infty}\frac{3x^{2}-2}{3x-2x^{2}} = \lim_{x\to_\infty}\frac{x^{2}\left(3-\frac{2}{x^{2}}\right)}{x^{2}\left(\frac{3}{x}-2\right)} = \lim_{x\to-\infty}\frac{3-\frac{2}{x^2}}{\frac{3}{x}-2} = -\frac{3}{2} 
\]

$ \lim_{\Delta t \to 0}\frac{s(t+\Delta t) + s(t)}{\Delta t} $ \&
$ \displaystyle{\lim_{\Delta t \to 0}\frac{s(t+\Delta t) + s(t)}{\Delta t}} $

%% 导数
% f^\prime(x)是标准写法,有时候可以写作f'(x)
\[ f^\prime(x) = \lim_{\Delta x \to 0}\frac{f(x+\Delta x)-f(x)}{\Delta x} \]
\[ f^\prime(x) = f'(x) = 15x^{4} + 6x^{2} \]
\[ \frac{\mathrm{d}}{\mathrm{d}x}f(x) = 15x^{4} + 6x^{2} \]
\[ \frac{\mathrm{d}^{2}}{\mathrm{d}x^{2}}f(x) = 60x^{3} + 12x \]

\[ f(x,y) = 3x^{5}y^{2} + 2x^{3}y + 1 \]
\[ \frac{\partial}{\partial x}f(x,y) = 15x^{4}y^{2} + 6x^{2}y \]
\[ \frac{\partial}{\partial y}f(x,y) = 6x^{5}y + 2x^{3} \]

\[ z = \mu\,\frac{\partial y}{\partial x}\bigg|_{x=0} \]

\newpage
%% 积分
\[ \int\frac{\mathrm{d}x}{\sqrt{a^{2}+x^{2}}} = \frac{1}{a} \arcsin\left(\frac{x}{a}\right) + C \]
\[ \int\tan^{2}x\,\mathrm{d}x = \tan x - x + C \]

\[ \int_{a}^{b}\left[\lambda_{1}f_{1}(x) + \lambda_{2}f_{2}(x)\right]\,\mathrm{d}x = \lambda_{1}\int_{a}^{b}f_{1}(x)\,\mathrm{d}x + \lambda_{2}\int_{a}^{b}f_{2}(x)\,\mathrm{d}x \]
\[ \int_{a}^{b}f(x)\,\mathrm{d}x = \int_{a}^{c}f(x)\,\mathrm{d}x + \int_{c}^{b}f(x)\,\mathrm{d}x \]

\[ \iint\limits_{D}f(x,y)\,\mathrm{d}\sigma \]
\[ \iiint\limits_{\Omega}\left(x^{2}+y^{2}+z^{2}\right)\,\mathrm{d}v \]

\begin{equation}
    \begin{aligned}
        V &= 2\pi\int_{0}^{1} x\left\{1-(x-1)^{2}\right\}\,\mathrm{d}x \\
        &= 2\pi\int_{0}^{2} \left\{-x^{3}+2x^{2}\right\}\,\mathrm{d}x \\
        &= 2\pi\left[-\frac{1}{4} x^{4} + \frac{2}{3} x^{3}\right]_{0}^{2} \\
        &= 8\pi/3
    \end{aligned}
\end{equation}

\newpage

%% ===== 线性代数 ======================================== %%
%% 矩阵
\[ \left[\begin{array}{ccc}
    1 & 2 & 3 \\
    4 & 5 & 6 \\ 
\end{array}\right] \]
\[ \left[\begin{array}{c|cc}
    1 & 2 & 3 \\ \hline
    4 & 5 & 6 \\ 
\end{array}\right] \]

\[ \begin{smallmatrix}
    1 & 2 & 3 \\
    4 & 5 & 6 \\ 
\end{smallmatrix} \]

\[ \begin{matrix}
    1 & 2 & 3 \\
    4 & 5 & 6 \\ 
\end{matrix} \]

\[ \begin{pmatrix}
    1 & 2 & 3 \\
    4 & 5 & 6 \\ 
\end{pmatrix} \]

\[ \begin{bmatrix}
    1 & 2 & 3 \\
    4 & 5 & 6 \\ 
\end{bmatrix}\]

\[ \begin{Bmatrix}
    1 & 2 & 3 \\
    4 & 5 & 6 \\ 
\end{Bmatrix} \]

\[ \begin{vmatrix}
    1 & 2 & 3 \\
    4 & 5 & 6 \\ 
\end{vmatrix} \]

\[ \begin{Vmatrix}
    1 & 2 & 3 \\
    4 & 5 & 6 \\ 
\end{Vmatrix} \]

\[ \mathbf{A} = \begin{bmatrix}
   a_{11} & a_{12} & \cdots & a_{1n} \\
   a_{21} & a_{22} & \cdots & a_{2n} \\
   \vdots & \vdots & \ddots & \vdots \\
   a_{m1} & a_{m2} & \cdots & a_{mn} 
\end{bmatrix}
\]

\[ \boldsymbol{y}:=\boldsymbol{y} + \left[\begin{array}{c|c|c} 
    A_{1}& \cdots & A_{n} \end{array}\right]
    \left[\begin{array}{c} 
    \boldsymbol{x}_{1} \\ \vdots \\ \boldsymbol{x}_{n} \\ \end{array}\right]
    = \boldsymbol{y} + \sum_{i=1}^{n}A_{i}\boldsymbol{x}_{i} \]

\[ \begin{bmatrix} 0 & 0 & 1 \\ 0 & 1 & 0 \\ 1 & 0 & 0 \\ \end{bmatrix}
\begin{bmatrix} a & b & c \\ b & d & e \\ c & e & f \\ \end{bmatrix}
= \begin{bmatrix} c & e & f \\ b & d & e \\ a & b & c \\ \end{bmatrix} \]

\newpage
%% 符号
\[ \mathbf{A}^{-1} \]   % 矩阵的逆
\[ \mathbf{A}^{+}, \mathbf{A}^{\dagger} \]  % 矩阵的伪逆 
\[ \mathbf{A}^{T}, \mathbf{A}^{\top} \]     % 矩阵的转置
\[ \mathbf{A}^{H} \]                        % 酉矩阵的转置
\[ \operatorname{rank}\left(\mathbf{A}\right) \]    % 矩阵的秩
\[ \operatorname{Tr}\left(\mathbf{A}\right) \]  % 矩阵的迹
\[ \det\left(\mathbf{A}\right) \]               % 矩阵的行列式

\[ \mathbf{A} \circ \mathbf{B} \]    % Hadamard 积(哈达玛积, 基本积)
\[ \mathbf{A} \otimes \mathbf{B} \]    % Kronecker 积(克罗内克积)
\begin{equation}
    \mathbf(X) \otimes \mathbf{Y} = \left[\begin{array}{cccc}
        x_{11}\mathbf{Y} & x_{12}\mathbf{Y} & \cdots & x_{1n}\mathbf{Y} \\
        x_{21}\mathbf{Y} & x_{22}\mathbf{Y} & \cdots & x_{2n}\mathbf{Y} \\
        \vdots & \vdots & \ddots & \vdots \\
        x_{m1}\mathbf{Y} & x_{m2}\mathbf{Y} & \cdots & x_{mn}\mathbf{Y} \\
    \end{array}\right]
\end{equation}

\vfill
%% 范数
\[ \left\|\mathbf{x}\right\|_{0} = \sqrt[0]{\sum_{i} x_{i}^{0}} \] % l0范数
\[ \left\|\mathbf{x}\right\|_{1} = \sum_{i}|x_{i}| \] % l1范数
\[ \left\|\mathbf{x}\right\|_{2} = \sqrt{\sum_{i} x_{i}^{2}} \] % l2范数

\[ \left\|\mathbf{X}\right\|_{F} \] % F范数
\[ \left\|\mathbf{X}\right\|_{*} \] % 核范数

\newpage
%% ===== 概率论与数理统计 ====================================== %%
% 贝叶斯公式
\[ p\left(\theta\mid y\right) = \frac{p\left(\theta,y\right)}{p\left(y\right)} = \frac{p\left(\theta\right)p\left(y\mid\theta\right)}{p\left(y\right)} \]
\[ p\left(\theta\mid y\right) \propto p\left(\theta\right) p\left(y\mid\theta\right) \]

\[ p\left(y\right) = \int p\left(y,\theta\right)\,d\theta = \int p\left(\theta\right)p\left(y\mid\theta\right)\,d\theta \]

% 期望与方差公式
\[ \mathbb{E}\left(x\right) = \int xp\left(x\right)\,\mathrm{d}x \]
\[ \mathbb{V}\left(x\right) = \int \left(x-\mathbb{E}\left(x\right)\right)^{2}p\left(x\right)\,\mathrm{d}x \]

% 正态分布
\[ x\sim\mathcal{N}\left(\mu, \sigma^{2}\right) \]
\[ p\left(x\right) = \frac{1}{\sqrt{2\pi}\sigma} \exp \left(-\frac{1}{2\sigma^{2}}\left(x-\mu\right)^2\right) \]

% 其他公式
\[ \theta\mid y\sim\operatorname{Gamma}\left(\alpha+\sum_{i=1}^{n}y_{i}, \beta+\sum_{i=1}^{n}x_{i}\right) \]
\begin{equation}
    \begin{aligned}
        y_{ij}&\sim\mathcal{N}\left(\alpha_{j}+x_{ij}\beta_{j}, \sigma_{y}^{2}\right) \\
        \begin{pmatrix} \alpha \\ \beta \end{pmatrix} & \sim\mathcal{N}\left(
            \begin{pmatrix} \mu_{\alpha} \\ \mu_{\beta} \end{pmatrix},
            \begin{pmatrix} \sigma_{\alpha}^2 & \rho\sigma_{\alpha}\sigma_{\beta} \\ \rho\sigma_{\alpha}\sigma_{\beta} & \sigma_{\beta}^{2} \end{pmatrix}
            \right)
    \end{aligned}
\end{equation}

\end{document}