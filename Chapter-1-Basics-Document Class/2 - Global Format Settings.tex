% %% 纸张方向和页边距
% \documentclass[12pt, b5paper]{article}

% \usepackage{lscape} % 纸张方向portrait(默认纵向), landscape(横向)
% \usepackage[margin=25mm]{geometry} % 调整页边距

% \begin{document}
% Hello, LaTeXers! This is our first LaTex document.

% \begin{landscape}
% Hello, LaTeXers! This is our first LaTex document.
% \end{landscape}

% \end{document}


%%%%%%%%%%%%%%%%%%%%%%%%%%%%%%%%%%%%%%%%%%%%%%%%%%%%%%%%%%%%%%%
%% 章节标题的字体格式
% | 编码   | 字体                        | 说明              |
% | :----- | :------------------------- | :---------------- |
% | `cmr`  | Computer Modern Roman      | LaTeX 默认罗马体   |
% | `cmss` | Computer Modern Sans Serif | 默认无衬线体       |
% | `cmtt` | Computer Modern Typewriter | 默认等宽体         |
% | `ptm`  | Times Roman                | Times 字体        |
% | `pvm`  | **Palatino Sans**          | Palatino 无衬线体  |
% | `phv`  | Helvetica                  | Helvetica 字体    |
% | `ppl`  | Palatino                   | Palatino 罗马体   |
% | `pbk`  | Bookman                    | Bookman 字体      |

% \documentclass{article}
% \usepackage{sectsty}
% \sectionfont{\fontfamily{phv} \fontseries{b}\fontsize{14pt}{20pt}\selectfont} % 一级标题字体格式设置
% \subsectionfont{\fontfamily{phv}\fontseries{b}\fontsize{12pt}{20pt}\selectfont} % 二级标题字体格式设置
% \subsubsectionfont{\fontfamily{phv}\fontseries{b}\fontsize{11pt}{20pt}\selectfont} % 三级标题字体格式设置
% % 字体族设置为 pvm(Palatino 字体的无衬线变体)
% % 字重(粗细)设置为 b = bold(粗体)
% % 字号11pt,行距20pt

% \title{LaTex cook-book}
% \author{author}
% \date{2026/02/16}  % \date{\today} 

% \begin{document}
% \maketitle
% \section{Introduction}
% \subsection{Sub Introduction}
% \subsubsection{Subsub Introduction}

% \paragraph{PA}
% Hello, LaTeXers! This is our first LaTex document.

% \subparagraph{Pa1}
% This document is our starting point fo learning LaTex and writing with it. It would not be difficult.
% \end{document}


%%%%%%%%%%%%%%%%%%%%%%%%%%%%%%%%%%%%%%%%%%%%%%%%%%%%%%%%%%%%%%%
%% 生成目录
%% \setcounter{tocdepth}{0} % 目录层次仅包括\part
%% \setcounter{tocdepth}{1} % 目录层次深入到\section
%% \setcounter{tocdepth}{2} % 目录层次深入到\subsection
%% \setcounter{tocdepth}{3} % 目录层次深入到\subsubsection

\documentclass[12pt]{article}
\begin{document}

% \setcounter{tocdepth}{1} 
\tableofcontents

\section{Section 1}
\subsection{1.1}
The text of 1.1
\subsection{1.2}
The text of 1.2
\subsection{1.3}
The text of 1.3

\section{Section 2}
\subsection{2.1}
The text of 2.1
\subsection{1.2}
The text of 2.2
\subsection{2.3}
The text of 2.3

\section*{Section 3} % 不会生成编号,也不会出现在目录中
\end{document}


