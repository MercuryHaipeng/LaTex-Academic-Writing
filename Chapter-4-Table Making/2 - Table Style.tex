\documentclass[12pt]{article}

\renewcommand{\arraystretch}{1.5} % 设置行高倍数
\usepackage{tabularx}
\usepackage{multirow} % 合并同列不同行的单元格
\usepackage[table]{xcolor}
\setlength{\arrayrulewidth}{2pt} % 修改表格线宽,默认0.4pt

\usepackage{booktabs} % 创建三线表格
\usepackage{longtable} % 创建跨页表格
\usepackage{rotating} % 旋转表格方向

\begin{document}
%% tabularx调整表格宽度
\begin{table}[htbp]
    \centering
    \label{table1}
    \begin{tabularx}{\textwidth}{|p{2cm}|p{3cm}|p{4cm}|p{3cm}|} 
        \hline
        Column1 & Column2 & Column3 & Column4 \\ \hline
        A1 & A2 & A3 & A4 \\ \hline
        B1 & B2 & B3 & B4 \\ \hline
        C1 & C2 & C3 & C4 \\ \hline
    \end{tabularx}
\end{table}

%% \multicolumn{合并列数}{合并后的列类型参数}{单元格内容}
%% \multirow{合并行数}{合并后的宽度,*为自动设置}{单元格内容}
%% 在同时合并多行多列的单元格时,除了第一行使用\multicolumn 和 \multirow 嵌套命令外,其余被合并的行均使用空内容的 \multirow 命令
\begin{table}[htbp]
    \centering
    \begin{tabular}{|l|l|l|l|} 
        \hline
        Column1 & Column2 & Column3 & Column4 \\ \hline
        \multicolumn{2}{|c|}{\multirow{2}{*}{A1, A2, B1 and B2}} & A3 & A4 \\
        \cline{3-4} % 创建一条从第3列到第4列的行分隔线
        \multicolumn{2}{|c|}{} & B3 & B4 \\ \hline
        C1 & C2 & C3 & C4 \\ \hline
    \end{tabular}
\end{table}


\newpage
%% =========================================================== %%
%% 插入彩色表格 \usepackage[table]{xcolor}
% 单元格填充和行填充颜色
\begin{table}[htbp]
    \centering
    \rowcolors{2}{red!30}{green!30} % 设置表格交替填充行颜色
    \begin{tabular}{|l|l|l|l|} 
        \hline
        Column1 & Column2 & Column3 & Column4 \\ \hline
        A1 & A2 & A3 & A4 \\ \hline
        B1 & B2 & B3 & \cellcolor{black!10}B4 \\ \hline % \cellcolor设置单元格颜色
        \rowcolor{yellow!50}C1 & C2 & C3 & C4 \\ \hline
        D1 & D2 & D3 & D4 \\ \hline
        E1 & E2 & E3 & E4 \\ \hline
    \end{tabular}
\end{table}

% 设置填充列颜色,注意行颜色会覆盖列颜色
\begin{table}[htbp]
    \centering
    \begin{tabular}{|>{\columncolor{blue!50}}l|>{\columncolor{blue!20}}l|>{\columncolor{pink!50}}l|>{\columncolor{black!20}}l|}
        \hline
        Column1 & Column2 & Column3 & Column4 \\ \hline
        A1 & A2 & A3 & A4 \\ \hline
        B1 & B2 & B3 & B4 \\ \hline 
        C1 & C2 & C3 & C4 \\ \hline
        D1 & D2 & D3 & D4 \\ \hline
        E1 & E2 & E3 & E4 \\ \hline
    \end{tabular}
\end{table}

%% =========================================================== %%
%% 创建跨页表格 \usepackage{longtable}
\begin{longtable}[c]{cccc}
    % 创建表格第一页的表头部分
    \caption{Title of a table} \\
    \hline
    Column1 & Column2 & Column3 & Column4 \\
    \hline
    \endfirsthead % \endfirsthead 之前的内容只会出现在表格第一页的表头部分
    % 创建表格除第一页之外的表头部分
    \caption{Title of a table - Continued} \\
    \hline
    Column1 & Column2 & Column3 & Column4 \\
    \hline
    \endhead % \endfirsthead 和 \endhead 之间的内容出现在表格除第一页之外的表头部分
    % 创建表格除了最后一页之外的表尾部分
    \hline
    \endfoot % \endhead 和 \endfoot 之间的内容出现在表格除最后一页之外的表尾部分
    % 创建表格最后一页的表尾部分
    \multicolumn{4}{c}{\textbf{End of table.}} \\
    \hline
    \endlastfoot % \endfoot 和 \endlastfoot 之间的内容只会出现在表格最后一页的表尾部分
    % 表格内容
    A1 & A2 & A3 & A4 \\
    B1 & B2 & B3 & B4 \\ 
    C1 & C2 & C3 & C4 \\
    D1 & D2 & D3 & D4 \\
    E1 & E2 & E3 & E4 \\
    A1 & A2 & A3 & A4 \\
    B1 & B2 & B3 & B4 \\ 
    C1 & C2 & C3 & C4 \\
    D1 & D2 & D3 & D4 \\
    E1 & E2 & E3 & E4 \\    
    A1 & A2 & A3 & A4 \\
    B1 & B2 & B3 & B4 \\ 
    C1 & C2 & C3 & C4 \\
    D1 & D2 & D3 & D4 \\
    E1 & E2 & E3 & E4 \\
    \hline
\end{longtable}

%% =========================================================== %%
%% 创建三线表格 \usepackage{booktabs}
\begin{table}[htbp]
    \centering
    \begin{tabular}{cccc}
        \toprule
        \multicolumn{2}{c}{\textbf{Type1}} & \\
        \cmidrule{1-2} 
        Column1 & Column2 & Column3 & Column4 \\
        \midrule
        A1 & A2 & A3 & A4 \\
        B1 & B2 & B3 & B4 \\ 
        C1 & C2 & C3 & C4 \\
        \bottomrule
    \end{tabular}
\end{table}

%% 表格逆时针旋转90度 \usepackage{rotating}
\begin{sidewaystable}[htbp]
    \centering
    \caption{Title of a table}
    \begin{tabular}{cccc}
        \toprule
        Column1 & Column2 & Column3 & Column4 \\
        \midrule
        A1 & A2 & A3 & A4 \\
        B1 & B2 & B3 & B4 \\ 
        C1 & C2 & C3 & C4 \\
        \bottomrule
    \end{tabular}
\end{sidewaystable}



\end{document}