\documentclass[12pt]{article}

\usepackage{booktabs}
\usepackage{threeparttable} 
\usepackage{diagbox} 
\usepackage{array}
\usepackage{tabularx}
\usepackage{tabulary}

\begin{document}

%% table 环境为创建的表格自动递增编号,并建立表格标题和索引标签
% 位置参数:h-当前位置,t-页面顶部,b-页面底部,p-下一页,!-若空间足够则强制放在指定位置;H-强制放在当前位置,需声明\usepackage{float}.
% tabular 环境中对齐方式:l, c, r; |: 单分隔线, ||: 双分隔线; &: 划分单元格; \hline: 创建行分隔线
Table~\ref{table1} shows the example of ABCD.
\begin{table}[htbp]
    \centering
    \caption{Example of A Table}
    \begin{tabular}{|l|c|c|c|}
        \hline
        Column1 & Column2 & Column3 & Column4 \\ \hline
        A1 & A2 & A3 & A4 \\ \hline
        B1 & B2 & B3 & B4 \\ \hline
        C1 & C2 & C3 & C4 \\ \hline
        D1 & D2 & D3 & D4 \\ \hline    
    \end{tabular}
    \label{table1}
\end{table}

%% 使用\fontsize调制表格内字体的大小 \usepackage{booktabs}
\begin{table}
    \fontsize{0.5cm}{0.8cm}\selectfont
    \centering
    \begin{tabular}{|c|c|c|c|}
        \hline
        Column1 & Column2 & Column3 & Column4 \\ \hline
        A1 & A2 & A3 & A4 \\ \hline
        B1 & B2 & B3 & B4 \\ \hline
        C1 & C2 & C3 & C4 \\ \hline
        D1 & D2 & D3 & D4 \\ \hline    
    \end{tabular}
\end{table}

\newpage
%% 插入表格注释 \usepackage{threeparttable}
\begin{table}
    \centering
    \begin{threeparttable}
    \begin{tabular}{|c|c|c|c|}
        \hline
        Column1\tnote{*} & Column2 & Column3 & Column4 \\ \hline
        A1\tnote{2} & A2 & A3 & A4 \\ \hline
        B1\tnote{3} & B2 & B3 & B4 \\ \hline
        C1 & C2 & C3 & C4 \\ \hline
    \end{tabular}
    \begin{tablenotes}
        \footnotesize
        \item[*] This is 1.
        \item[2] This is a remark example.
        \item[3] This is another remark example and with a very long content, but the contents will be wrapped.
    \end{tablenotes}
    \end{threeparttable}
\end{table}

%% 插入各类斜线 \usepackage{diagbox}
\begin{table}[htbp]
    \centering
    \caption{Example of ABC}
    \begin{tabular}{|c|c|c|c|}
        \hline
        \diagbox[width=5em]{$A$}{$B$} & Column2 & Column3 & Column4 \\ \hline
        A1 & A2 & A3 & A4 \\ \hline
        B1 & B2 & B3 & B4 \\ \hline
        C1 & C2 & C3 & C4 \\ \hline
        D1 & D2 & D3 & D4 \\ \hline    
    \end{tabular}
\end{table}

\begin{table}[htbp]
    \centering
    \caption{Example of ABC}
    \begin{tabular}{|c|c|c|c|}
        \hline
        \diagbox[width=5em]{$A$}{$B$}{$C$} & Column2 & Column3 & Column4 \\ \hline
        A1 & A2 & A3 & A4 \\ \hline
        B1 & B2 & B3 & B4 \\ \hline
        C1 & C2 & C3 & C4 \\ \hline
        D1 & D2 & D3 & D4 \\ \hline    
    \end{tabular}
\end{table}

\newpage
%% ================================================ %%
%% 文本对齐和自动换行 \usepackage{array}
% p{列宽}:顶部基线对齐  m{列宽}:纵向居中基线对齐  b{列宽}:底部基线对齐
% >{\raggedright\arraybackslash}: 一列单元格内容设置为左对齐  
% >{\centering\arraybackslash}: 一列单元格内容设置为横向居中对齐
% >{\raggedleft\arraybackslash}: 一列单元格内容设置为右对齐
\begin{table}[htbp]
    \centering
    \caption{Title of a table}
    \label{first label}
    \begin{tabular}{|>{\raggedright\arraybackslash}m{2.3cm}|>{\centering\arraybackslash}p{2.3cm}|>{\centering}b{2.3cm}|>{\raggedleft\arraybackslash}m{2.3cm}|}
        \hline
        Column1 & Column2 Column2 & Column3 Column3 Column3 & Column4 Column4 Column4 Column4 \\ \hline
        Value1 & Value2 Value2 & Value3 Value3 Value3 & Value4 Value4 Value4 Value4 \\ \hline
        Value1 & Value2 Value2 & Value3 Value3 Value3 & Value4 Value4 Value4 Value4 \\ \hline
    \end{tabular}
\end{table}

%% tabularx环境中,将需要自动换行的列类型参数设为X
\begin{table}[htbp]
    \centering
    \caption{Title of a table}
    \label{second label}
    \begin{tabularx}{\linewidth}{|>{\raggedleft\arraybackslash}X|X|X|>{\centering\arraybackslash}X|} 
        \hline
        Column1 & Column2 & Column3 & Column4 \\ \hline
        Value1 & Value2 Value2 & Value3 Value3 Value3 & Value4 Value4 Value4 Value4 \\ \hline
        This is Value1. This is Value1. & This is Value2. This is Value2. & This is Value3. This is Value3. &This is Value4. This is Value4. \\ \hline
    \end{tabularx}
\end{table}

%% tabulary环境自动换行,L(左对齐), C(居中对齐), R(右对齐)
\begin{table}[htbp]
    \centering
    \caption{Title of a table}
    \label{third label}
    \begin{tabulary}{\linewidth}{|L|C|C|R|} 
        \hline
        Column1 & Column2 & Column3 & Column4 \\ \hline
        Value1 & Value2 Value2 & Value3 Value3 Value3 & Value4 Value4 Value4 Value4 \\ \hline
        This is Value1. This is Value1. & This is Value2. This is Value2. & This is Value3. This is Value3. &This is Value4. This is Value4. \\ \hline
    \end{tabulary}
\end{table}

\end{document}