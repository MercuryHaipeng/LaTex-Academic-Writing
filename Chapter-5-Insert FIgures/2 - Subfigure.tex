\documentclass[12pt]{article}
\usepackage{graphicx}
\usepackage{subcaption}
\usepackage{amssymb}
\usepackage{amsmath}
\usepackage{wrapfig} % 设置图片为文字环绕

\begin{document}
\section{Brief Introduction}
Figure \ref{fig:fig1} contains sub-figure \ref{subfig:subfig1}, sub-figure \ref{subfig:subfig2} and sub-figure \ref{subfig:subfig3}.

\begin{figure}[!htbp]
    \centering
    % 插入第一张子图,同时设置子图环境宽度占页面宽度的比例
    \begin{subfigure}{0.3\linewidth} 
        \centering
        % 设置图片宽度占当前子图环境宽度的比例
        \includegraphics[width=0.6\linewidth]{Figures/fig1.jpg}
        \caption{A lovely child.}
        \label{subfig:subfig1}   
    \end{subfigure}
    % 插入第二张子图
    \begin{subfigure}{0.3\linewidth}
        \centering
        \includegraphics[width=0.6\linewidth]{Figures/fig2.jpg}
        \caption{A sailing competitor.}
        \label{subfig:subfig2}   
    \end{subfigure}
    % 插入第三张子图
    \begin{subfigure}{0.3\linewidth}
        \centering
        \includegraphics[width=0.6\linewidth]{Figures/fig3.jpg}
        \caption{A javelin thrower.}
        \label{subfig:subfig3}   
    \end{subfigure}
\caption{Three figures with different characters. }
\label{fig:fig1}
\end{figure}


%% ===================================================== %% 
\section{Adjust the Subfigures Spacing}
% - \hfill: 对位于相同行的子图实现多个子图横向等距分布
% - \hspace{横向距离}: 定制横向图片距离。负值可产生图片重叠的效果
% - \quad, \qquad: 设置不同预设长度的横向图片距离
% - \vfill: 实现图片纵向等距分布
The horizontal space among Sub-figures in figure \ref{fig:fig2} is controlled by $\backslash$\textit{hfill}.
\begin{figure}[!htbp]
    \centering
    % 插入第一张子图,同时设置子图环境环境宽度占页面宽度的比例
    \begin{subfigure}{0.3\linewidth} 
        \centering
        % 设置图片宽度占当前子图环境宽度的比例
        \includegraphics[width=0.6\linewidth]{Figures/fig1.jpg}
        \caption{A lovely child.}
    \end{subfigure}
    \hfill
    % 插入第二张子图
    \begin{subfigure}{0.3\linewidth}
        \centering
        \includegraphics[width=0.6\linewidth]{Figures/fig2.jpg}
        \caption{A sailing competitor.}
    \end{subfigure}
    \hfill
    % 插入第三张子图
    \begin{subfigure}{0.3\linewidth}
        \centering
        \includegraphics[width=0.6\linewidth]{Figures/fig3.jpg}
        \caption{A javelin thrower.}
    \end{subfigure}
\caption{Three figures with different characters. }
\label{fig:fig2}
\end{figure}

The horizontal space among Sub-figures in figure \ref{fig:fig3} is controlled by $\backslash$\textit{space}.
\begin{figure}[!http]
    \centering
    % 插入第一张子图,同时设置子图环境环境宽度占页面宽度的比例
    \begin{subfigure}{0.3\linewidth} 
        \centering
        % 设置图片宽度占当前子图环境宽度的比例
        \includegraphics[width=0.6\linewidth]{Figures/fig1.jpg}
    \end{subfigure}
    \hspace{-3cm}
    % 插入第二张子图
    \begin{subfigure}{0.3\linewidth}
        \centering
        \includegraphics[width=0.6\linewidth]{Figures/fig2.jpg}
    \end{subfigure}
    \hspace{-3cm}
    % 插入第三张子图
    \begin{subfigure}{0.3\linewidth}
        \centering
        \includegraphics[width=0.6\linewidth]{Figures/fig3.jpg}
    \end{subfigure}
\caption{Three figures with different characters. }
\label{fig:fig3}
\end{figure}

\begin{figure}[!http]
    \centering
    % 插入第一张子图,同时设置子图环境环境宽度占页面宽度的比例
    \begin{subfigure}{0.5\linewidth} 
        \centering
        % 设置图片宽度占当前子图环境宽度的比例
        \includegraphics[width=0.6\linewidth]{Figures/fig1.jpg}
    \end{subfigure}
    \vfill
    % 插入第二张子图
    \begin{subfigure}{0.5\linewidth}
        \centering
        \includegraphics[width=0.6\linewidth]{Figures/fig2.jpg}
    \end{subfigure}
    \vfill
    % 插入第三张子图
    \begin{subfigure}{0.5\linewidth}
        \centering
        \includegraphics[width=0.6\linewidth]{Figures/fig3.jpg}
    \end{subfigure}
\caption{Three figures with different characters.}
\label{fig:fig4}
\end{figure}


%% ============================================== %%
% 图片并排
\section{Reformat the Layout}
The two figures are displayed side by side.
\begin{figure}[htbp]
    \centering
    \begin{minipage}[t]{0.48\linewidth}
        \centering
        \includegraphics[width=6cm]{Figures/fig1.jpg}
        \caption{Child-1.}
    \end{minipage}
    \begin{minipage}[t]{0.48\linewidth}
        \centering
        \includegraphics[width=6cm]{Figures/fig1.jpg}
        \caption{Child-2.}
    \end{minipage}
\end{figure}

\newpage
% 设置图片文字环绕,需要宏包\usepackage{wrapfig}
There is a loverly child who is playing in the living room.
\begin{wrapfigure}{r}{8cm}
    \centering
    \includegraphics[width=0.4\textwidth]{Figures/fig1.jpg} % 图片宽度固定不变
    % \includegraphics[width=0.4\linewidth]{Figures/fig1.jpg} % 图片宽度随环境变化自动缩放
    \caption{A lovely child.}
    \label{fig:child}
\end{wrapfigure}

In descriptive statistics, a box plot or boxplot is a method for graphically depicting groups of numerical data through their quartiles. Box plots may also have lines extending from the boxes (whiskers) indicating variability outside the upper and lower quartiles, hence the terms box-and-whisker plot and box-and-whisker diagram. Outliers may be plotted as individual points. Box plots are non-parametric: they display variation in samples of a statistical population without making any assumptions of the underlying statistical distribution (though Tukey's boxplot assumes symmetry for the whiskers and normality for their lentgh.)



\end{document}