\documentclass[12pt]{article}
\usepackage{parskip}  % 取消缩进,增加段间距

\usepackage{amsmath}    % 数学公式排版
\usepackage{amssymb}    % 数学符号扩展 
\usepackage{amsfonts}   % 数学专用字体

\begin{document}
%% 行内公式和行间公式
$ x+y=2 $ is a simple quadratic equation.
$$ x+y=2 $$

%% ================================================== %%
%% equation 环境
% 自动带有编号的公式
\begin{equation}\label{eq1}
    x+y=2
\end{equation}
Equation~\eqref{eq1} shows a simple formula.

% 移除公式编号
\begin{equation*}\label{eq2}
    x+y=2
\end{equation*}

\newpage
%% ================================================== %%
%% align 环境
\begin{align}
    x+y=2 \\
    2x+y=3
\end{align}

\begin{align*}
    2x+1&=7 & 3y-2&=-5 & -5z+8&=-2 \\
    2x&=6 & 3y&=-3 & -5z&=-10 \\
    x&=3 & y&=-1 & z&=2
\end{align*}

\begin{align}
    2x+1&=7 & 3y-2&=-5 & -5z+8&=-2 \nonumber \\ % 某一行不编号
    2x&=6 & 3y&=-3 & -5z&=-10 \\
    x&=3 & y&=-1 & z&=2 
\end{align}

% 多行公式共用同一个编号
\begin{equation}
    \begin{aligned}
        2x+1&=7 & 3y-2&=-5 & -5z+8&=-2 \\
        2x&=6 & 3y&=-3 & -5z&=-10 \\
        x&=3 & y&=-1 & z&=2
    \end{aligned}
\end{equation}

% 自动居左对齐公式环境
\begin{flalign}
    &x+y=2 & \\
    &2x+y=3
\end{flalign}

\newpage
%% ================================================== %%
%% gather 环境
\begin{gather}
    x+y=2 \\
    2x+y=3
\end{gather}

\begin{gather*}
    x+y=2 \\
    2x+y=3
\end{gather*}

\newpage
%% ================================================== %%
%% 字符类型
$ \boldsymbol{X, Y, Z, x, y, z} $ \\    % 数学粗斜体
$ \mathrm{X, Y, Z, x, y, z} $ \\        % 罗马体(正体/直立体)
$ \mathit{X, Y, Z, x, y, z} $ \\        % 数学斜体
$ \mathbf{X, Y, Z, x, y, z} $ \\        % 粗罗马体
$ \mathsf{X, Y, Z, x, y, z} $ \\        % 无衬线体
$ \mathtt{X, Y, Z, x, y, z} $ \\        % 等宽字体(打字机字体)
$ \mathcal{X, Y, Z} $ \\                % 花体(手写书法体)大写字母
$ \mathbb{X, Y, Z} $ \\                 % 黑板体(空心粗体) 大写字母
$ \mathfrak{X, Y, Z, x, y, z} $ \\      % 哥特体/德文体

\boldmath   % 加粗特定公式
\begin{equation}
    x^{2} + y^{2} - \sin z = 4
\end{equation}
\unboldmath

%% 调整公式大小
$ \displaystyle{f(x)=\sum_{i=1}^{n}\frac{1}{x_{i}}} $, \\
$ \textstyle{f(x)=\sum_{i=1}^{n}\frac{1}{x_{i}}} $, \\
$ \scriptstyle{f(x)=\sum_{i=1}^{n}\frac{1}{x_{i}}} $, \\
$ \scriptscriptstyle{f(x)=\sum_{i=1}^{n}\frac{1}{x_{i}}} $. 

% 限定字符区域调整公式大小
\begingroup
\small
\begin{align}
    x+y=2 \\
    2x+y=3
\end{align}
\endgroup

\begingroup
\Large
\begin{align}
    x+y=2 \\
    2x+y=3
\end{align}
\endgroup

\newpage
%% ================================================== %%
%% array 环境: l(左对齐), c(居中对齐), r(右对齐)
\begin{equation}
    \left\{\begin{array}{l} % 左侧显示大括号,右侧占位但不显示符号
        x+y=2 \\
        2x+y=3
    \end{array}\right.
\end{equation}

\begin{align}
    \left\{\begin{array}{r}
        x+y=2 \\
        2x+y=3
    \end{array}\right.
\end{align}

\begin{equation}
    \begin{array}{c@{\qquad}c}  % 超大间距的两列居中数组
        A = B + C \qquad \Rightarrow & D = E - F, \\ \\
        G = H \qquad \Rightarrow & K = P + Q + M.
    \end{array}
\end{equation}

\end{document}