%% 使用 ctex 宏包或 ctexart
%% 英文标准文档类 article,通过 ctex 宏包添加中文支持
% \documentclass{article}
% \usepackage[UTF8]{ctex}   

%% 专用中文文档类 ctexart,内置完整中文支持
% \documentclass{ctexart}

% %% 以下两种方式实现页码下方居中显示
% % \usepackage{fancyhdr}
% % % 清除默认样式
% % \pagestyle{fancy}
% % \fancyhf{}  % 清空页眉页脚
% % % 设置页脚:页码居中
% % \fancyfoot[C]{\thepage}
% % % 可选:去掉页眉横线
% % \renewcommand{\headrulewidth}{0pt}

% \pagestyle{plain}  % 页码在底部居中,无页眉
% %%

% \begin{document}

% 江南好,风景旧曾谙。日出红花不胜火,春来江水绿如蓝。能不忆江南?

% 青青园中葵,朝露待日晞。 
% 阳春布德泽,万物生光辉。 
% 常恐秋节至,焜黄华叶衰。 
% 百川东到海,何时复西归? 
% 少壮不努力,老大徒伤悲!

% \end{document}

%%%%%%%%%%%%%%%%%%%%%%%%%%%%%%%%%%%%%%%%%%%%%%%%%%%%%%%%%%%%%
%% 楷书:\kaishu, 宋体:\songti, 黑体:\heiti, 仿宋:\fangsong 
\documentclass{ctexart}
\begin{document}

{\kaishu 【楷书】\\
江南好,风景旧曾谙。日出红花不胜火,春来江水绿如蓝。能不忆江南?

青青园中葵,朝露待日晞。 
阳春布德泽,万物生光辉。 
常恐秋节至,焜黄华叶衰。 
百川东到海,何时复西归? 
少壮不努力,老大徒伤悲!}

{\heiti 【黑体】\\
江南好,风景旧曾谙。日出红花不胜火,春来江水绿如蓝。能不忆江南?

青青园中葵,朝露待日晞。 
阳春布德泽,万物生光辉。 
常恐秋节至,焜黄华叶衰。 
百川东到海,何时复西归? 
少壮不努力,老大徒伤悲!}

{\fangsong 【仿宋】\\
江南好,风景旧曾谙。日出红花不胜火,春来江水绿如蓝。能不忆江南?

青青园中葵,朝露待日晞。 
阳春布德泽,万物生光辉。 
常恐秋节至,焜黄华叶衰。 
百川东到海,何时复西归? 
少壮不努力,老大徒伤悲!}

{\songti 【宋体】\\
江南好,风景旧曾谙。日出红花不胜火,春来江水绿如蓝。能不忆江南?

青青园中葵,朝露待日晞。 
阳春布德泽,万物生光辉。 
常恐秋节至,焜黄华叶衰。 
百川东到海,何时复西归? 
少壮不努力,老大徒伤悲!}

\end{document}

%%%%%%%%%%%%%%%%%%%%%%%%%%%%%%%%%%%%%%%%%%%%%%%%%%%%%%%%%%%%%%%
%% 其他宏包
% \usepackage{xeCJK} % % 专门用于编译中文, ctexart自带, article需要手动加载
% \usepackage{CJKutf8} % 一般在 pdfLaTeX 编译器中使用
