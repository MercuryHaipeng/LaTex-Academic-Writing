% %% 调整字体样式和大小
% \documentclass[12pt]{article}

% \usepackage{parskip}  % 自动取消缩进,增加段间距
% % \usepackage{indentfirst} % 章节后第一段首行缩进

% \begin{document}
% %% 字体样式调整
% Produce \textit{italicized} text. \\   % 生成斜体文字
% Produce \textbf{bold face} text. \\  % 生成粗体文字
% Produce \texttt{typewriter font} text. \\  % 生成打字机字体文字
% Produce \textsc{small caps} text. \\ % 生成小型大写字母文字

% \vspace{1em}
% %% 英文字母全部改为大写
% \uppercase{Use uppercase command to force all uppercase.} \\
% \MakeUppercase{Use MakeUppercase command to force all uppercase.}

% \newpage

% %% 调整字体大小
% Produce {\tiny tiny word} \\
% Produce {\scriptsize script size word} \\
% Produce {\footnotesize footnote size word} \\
% Produce {\normalsize normal size word} \\
% Produce {\large large word} \\
% Produce {\Large Large word} \\
% Produce {\LARGE LARGE word} \\
% Produce {\huge huge word} \\
% Produce {\Huge Huge word} \\

% \vspace{1em}
% % 也可以用环境来控制字体
% Produce \begin{large} large word \end{large} \\
% Produce \begin{Large} Large word \end{Large} \\
% Produce \begin{LARGE} large word \end{LARGE} \\

% \end{document}

%% ======================================== %%
%% 字体颜色,下划线和删除线
% \documentclass[12pt]{article}

% \usepackage{parskip}  % 自动取消缩进,增加段间距

% \usepackage{color}
% \definecolor{kugreen}{RGB}{50, 93, 61}
% \definecolor{kugreenlys}{RGB}{132, 158, 139}
% \definecolor{kugreenlyslys}{RGB}{173, 190, 177}
% \definecolor{kugreenlyslyslys}{RGB}{214, 233, 216}

% \usepackage[T1]{fontenc}
% \usepackage{palatino}   % palatino 宏包提供了 Palatino 字体

% \usepackage{ulem}   % 设置下划线
% \usepackage{soul}   % 设置删除线

% \begin{document}
% % 字体颜色设置
% This is a simple example for using \LaTeX. \\
% {\color{kugreen}This is a simple example for using \LaTeX.} \\
% {\color{kugreenlys}This is a simple example for using \LaTeX.} \\
% {\color{kugreenlyslys}This is a simple example for using \LaTeX.} \\
% {\color{kugreenlyslyslys}This is a simple example for using \LaTeX.} \\

% \newpage

% % 下划线与删除线
% Generate \underline{underlined} text. \\    % 生成带下划线的文本
% Generate \uline{underlined} text. \\    % 生成单下划线的文本
% Generate \uuline{double underlined} text. \\    % 生成双下划线的文本
% Generate \uwave{wavy underlined} text. \\   % 生成波浪线的文本

% This is text\_with\_underscores.    % 文本下标绘制下划线

% Generate \st{many strikethrough} text. \\   % 生成带删除线的文本


% \end{document}

%% ======================================== %%
%% 特殊字符
\documentclass[12pt]{article}
\usepackage{pifont}
\begin{document}
How to write a number in an article?

Type 1: \ding{172}-\ding{173}-\ding{174}-\ding{175}-\ding{176}-\ding{177}-\ding{178}-\ding{179}-\ding{180}-\ding{181}

Type 2: \ding{182}-\ding{183}-\ding{184}-\ding{185}-\ding{186}-\ding{187}-\ding{188}-\ding{189}-\ding{190}-\ding{191}

Type 3: \ding{192}-\ding{193}-\ding{194}-\ding{195}-\ding{196}-\ding{197}-\ding{198}-\ding{199}-\ding{200}-\ding{201}

Type 4: \ding{202}-\ding{203}-\ding{204}-\ding{205}-\ding{206}-\ding{207}-\ding{208}-\ding{209}-\ding{210}-\ding{211}

\end{document}

