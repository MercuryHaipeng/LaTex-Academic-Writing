%% 1. 无序列表
\documentclass[12pt]{article}

% \usepackage{amssymb} % 引入数学宏包
% \renewcommand\labelitemi{$\blacksquare$} % 条目符号设置为黑色方块
% \renewcommand\labelitemii{$\square$} % 条目符号设置为空心方块

\usepackage{enumitem} % 调整列表上下左右的缩进间距

\begin{document}
% % 一般无序列表
% \begin{itemize}
%     \item Python    % 起始符号为圆点
%     \item LaTeX     % 起始符号为圆点
%     \item[*] GitHub % 起始符号为星号
% \end{itemize}

% 二级列表:方块嵌套展示
\begin{itemize}
    \item Languages
        \begin{itemize}
            \item Python
            \item Java
        \end{itemize}
    \item Tools
        \begin{itemize}
            \item LaTeX
            \item GitHub
        \end{itemize}
\end{itemize}

\newpage
%% ========================================================= %%
%% 2. 有序列表
\begin{enumerate}
    \item pencil
    \item calculator
    \item ruler
    \item notebook
    \begin{enumerate}
        \item notes
        \begin{enumerate}
            \item note A
            \begin{enumerate}
                \item note a
            \end{enumerate}
            \item note B
        \end{enumerate}
        \item homework
        \item assessments
    \end{enumerate}
\end{enumerate}

\newpage
%% ========================================================= %%
%% 3. 阐述性列表
\begin{description}
    \item[CNN] Convolutional Neural Networks
    \item[RNN] Recurrent Neural Network
    \item[CRNN] Convolutional Recurrent Neural Network   
\end{description}

\newpage
%% ========================================================= %%
%% 4. 自定义列表格式
%% topsep: 列表环境与上下文的距离  parsep: 条目内段落之间的距离
%% itemsep: 条目之间的距离  partopsep: 条目与下面段落的距离
%% leftmargin: 列表环境左边的空白长度  rightmargin: 列表环境右边的空白长度
%% labelsep: 标号与列表环境左侧的距离  itemindent: 条目的缩进距离
%% labelwidth: 调整标号的宽度  listparindent: 条目下面段落的缩进距离
Default spacing:
\begin{itemize}
    \item Python
    \item LaTeX  
    \item GitHub
\end{itemize}

Custom Spacing:
\begin{itemize}[itemsep=15pt, topsep=20pt, itemindent=20pt, labelsep=20pt]
    \item Python
    \item LaTeX  
    \item GitHub
\end{itemize}



\end{document}