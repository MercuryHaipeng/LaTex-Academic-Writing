\documentclass{article}

% \csvautotabular{}基于 csv 文件快速创建表格
\usepackage{csvsimple}

% \csvautobooktabular{}基于 csv 文件快速创建三线表格
\usepackage{booktabs} 

%% \csvreader[属性设置]{csv文件名或文件路径}{定义数据列名}{需要导入的数据列名},属性设置包括:
% tabular: 定义列类型。列类型个数应与需要导入的列数一致
% table head: 定义表头,包括标题行的顶线、列名及底线
% late after line: 定义行分隔线

\begin{document}

%% 读取csv文件创建表格
\begin{table}
    \centering
    \caption{A table imported from csv file}
    \label{label of table1}
    \csvautotabular{grade.csv}
\end{table}

% 读取CSV文件创建三线表格
\begin{table}
    \centering
    \caption{A table imported from csv file}
    \label{label of table2}
    \csvautobooktabular{grade.csv}
\end{table}

%% 设置表格属性
\begin{table}
    \centering
    \caption{A table imported from csv file}
    \label{label of table3}
    \csvreader[
        tabular = |r|l|c|c|c|, % 定义列类型,个数与需要导入的列数一致
        table head = \hline & Person & Matr.~No. & Gender & Grade \\ \hline, % 定义表头,包括标题行的顶线、列名和底线
        late after line = \\ \hline % 定义行分隔线
        ]
        {grade.csv} % csv 文件名
        {name=\name, givenname=\firstname, matriculation=\matnumber, gender=\gender, grade=\grade} % 定义数据列名
        {\thecsvrow & \firstname~\name & \matnumber & \gender & \grade}
\end{table}

\begin{table}
    \centering
    \caption{A table imported from csv file}
    \label{label of table3}
    \csvreader[
        tabular = |r|l|c|, % 定义列类型,个数与需要导入的列数一致
        table head = \hline & Person & Matr.~No. \\ \hline, % 定义表头,包括标题行的顶线、列名和底线
        late after line = \\ \hline % 定义行分隔线
        ]
        {grade.csv} % csv 文件名
        {name=\name, givenname=\firstname, matriculation=\matnumber, gender=\gender, grade=\grade} % 定义数据列名
        {\thecsvrow & \firstname~\name & \matnumber}
\end{table}

\end{document}